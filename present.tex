% Copyright 2007 by Till Tantau
%
% This file may be distributed and/or modified
%
% 1. under the LaTeX Project Public License and/or
% 2. under the GNU Public License.
%
% See the file doc/licenses/LICENSE for more details.



\documentclass{beamer}

%
% DO NOT USE THIS FILE AS A TEMPLATE FOR YOUR OWN TALKS�!!
%
% Use a file in the directory solutions instead.
% They are much better suited.
%


% Setup appearance:

\usetheme{Darmstadt}
%\usetheme{CambridgeUS}
%\usetheme{Berkeley}
%\usetheme{Hannover}
\usefonttheme[onlylarge]{structurebold}
\setbeamerfont*{frametitle}{size=\normalsize,series=\bfseries}
\setbeamertemplate{navigation symbols}{}
\setbeamercovered{transparent=30}
\setbeamertemplate{footline}{
    
    \leavevmode%
    \hbox{%
    \begin{beamercolorbox}[wd=.4\paperwidth,ht=2.25ex,dp=1ex,center]{author in head/foot}%
        \usebeamerfont{author in head/foot}\insertshortauthor
    \end{beamercolorbox}%

    \begin{beamercolorbox}[wd=.6\paperwidth,ht=2.25ex,dp=1ex,center]{title in head/foot}%
        \usebeamerfont{title in head/foot}\insertshorttitle\hspace*{3em}
        \insertframenumber{} / \inserttotalframenumber\hspace*{1ex}
    \end{beamercolorbox}}%
    \vskip0pt% 
    
    
}

% Standard packages

\usepackage[english]{babel}
\usepackage[latin1]{inputenc}
\usepackage{times}
\usepackage[T1]{fontenc}
\usepackage{graphicx}
%\usepackage{natbib}
\usepackage{enumerate}



% Setup TikZ

\usepackage{tikz}
\usetikzlibrary{arrows}
\tikzstyle{block}=[draw opacity=0.7,line width=1.4cm]


% Author, Title, etc.

\title[ 2014 Fall Group Meeting ] 
{%
    High-Sensitivity Hardware Trojan Detection Using
    Multimodal Characterization
  %
}

\author[Boyou Zhou]
{
  Kangqiao Hu\inst{1}, Nazma Nowrozy\inst{2}, Sherief Reday \inst{2}\\
  and Farinaz Koushanfar\inst{1}\\
  \textit{Presented by Boyou Zhou\inst{*}}
}

\institute[Boston University, MA]
{
  \inst{1} ECE Department, Rice University, Houston, Texas, 77005 \\
  Email: kangqiao.hu, farinaz@rice.edu\\
  \inst{2} School of Engineering, Brown University, Providence, RI, 02906 \\
  Email: abdullah nowroz, sherief reda@brown.edu\\
  \inst{*} Boston Univeristy, MA
}

\date[Modified on \today]
%{Created on Oct 16 2014, Modified on \today}

% The main document

\begin{document}

\begin{frame}
  \titlepage
\end{frame}

\begin{frame}{Outline}
  \tableofcontents
\end{frame}

\AtBeginSubsection{
    \frame<beamer>{ 
    \frametitle{Outline}   
    \tableofcontents[currentsection,currentsubsection]
    }
}

%%   * -.-.-.-.-.-.-.-.-.-.-.-.-.-.-.-.-.-.-.-.
%    
%       *   File Name : Intro.tex
%       *   Purpose :
%       *   Creation Date : 10-16-2014
%       *   Last Modified : Thu 16 Oct 2014 09:21:29 PM EDT
%       *   Created By : Boyou Zhou
%    
%    The MIT License (MIT)
%    
%    Copyright (c) 2014 Boyou Zhou              
%    
%    Permission is hereby granted, free of charge, to any person obtaining a copy
%    of this software and associated documentation files (the "Software"), to deal
%    in the Software without restriction, including without limitation the rights
%    to use, copy, modify, merge, publish, distribute, sublicense, and/or sell
%    copies of the Software, and to permit persons to whom the Software is
%    furnished to do so, subject to the following conditions:
%    
%    The above copyright notice and this permission notice shall be included in all
%    copies or substantial portions of the Software.
%    
%    THE SOFTWARE IS PROVIDED "AS IS", WITHOUT WARRANTY OF ANY KIND, EXPRESS OR
%    IMPLIED, INCLUDING BUT NOT LIMITED TO THE WARRANTIES OF MERCHANTABILITY,
%    FITNESS FOR A PARTICULAR PURPOSE AND NONINFRINGEMENT. IN NO EVENT SHALL THE
%    AUTHORS OR COPYRIGHT HOLDERS BE LIABLE FOR ANY CLAIM, DAMAGES OR OTHER
%    LIABILITY, WHETHER IN AN ACTION OF CONTRACT, TORT OR OTHERWISE, ARISING FROM,
%    OUT OF OR IN CONNECTION WITH THE SOFTWARE OR THE USE OR OTHER DEALINGS IN THE
%    SOFTWARE.
%    
%_._._._._._._._._._._._._._._._._._._._._.*/

%\begin{frame}{Three Parts of Work}
%%    Picture Example
%%    \begin{figure}[p]
%%        \includegraphics[width=4in]{../doc_image/presentation/radio_freq.png}
%%        \caption{Spectrum of electromagnetic waves}
%%        \label{Spectrum of electromagnetic waves}
%%    \end{figure}
%    \begin{itemize}
%        \item  \textit{Sheng Wei}
%            \begin{itemize}
%\pause
%                \item   The main point of the paper is to develop an one-gate trojan trigger 
%                circuit. The attack circuitry is on TrustHub. TrustHub has various kinds
%                of attacking circtuits' verilog codes.
%\pause
%                \item   Sheng works on only a small part of the project for developing triggers.
%                He can help us develop the trigger but not the attacking circtuit.
%\pause
%                \item   Leakage power analysis has been applied for detecting trojans that are 
%                off at first and then turned on later. For example, trigger can comes from a
%                counter.
%            \end{itemize}
%    \end{itemize}
%\end{frame}



\section{Introduction}
\subsection{How Hardware Can Be Attacked?}
\begin{frame}{Method of Hardware Attacking}
    \begin{columns}

        \column{0.5\textwidth}
        \begin{figure}[p]
            \includegraphics[width=2in]{../img/HT_method.PNG}
            \caption{\textbf{Hardware Trojan Attacks by Different Parties at Different
                    stages of IC Life Cycles}}
            \label{Hardware Trojan Attack Method}
        \end{figure}

        \column{0.5\textwidth}
        \begin{itemize}
\pause
            \item Insert Trojan in IP.
\pause
            \item Insert HT in Design.
\pause
            \item \textbf{Insert HT After Reverse-engineered the GDS.}
        \end{itemize}

    \end{columns}
\end{frame}

\subsection{What kind of HT Are We Interested in?}
\begin{frame}{Detecting Hardware Trojans that has been Insterted After Fab}
\pause
    \begin{block}{Detectability of Hardware Trojan}
        \begin{itemize}
            \item \textit{Size is Small.} 
            \item \textit{Side-Channel Leakage is negligible.}
            \item \textit{Little chance for hardware to be Trigger}\\
                          Avoid extensive functional testing.
        \end{itemize}
    \end{block}

\pause
    \begin{block}{Side Channel Analysis}
        \begin{itemize}
            \item \textit{Power Consumption}
            \item \textit{Electromagnetic Radiation}
            \item \textit{Timing Behavior}
            \item \textit{Sound Waves}
        \end{itemize}
    \end{block}
\end{frame}

\section{Thermal Analysis}
\subsection{Basic Idea}

\begin{frame}{Thermal Map}
    \begin{columns}
        \column{0.5\textwidth}
        \begin{figure}[p]
            \includegraphics[width=2in]{../img/basic_idea.PNG}
            \caption{\textbf{Thermal Map of MIPS}}
            \label{Hardware Trojan Attack}
        \end{figure}

        \column{0.5\textwidth}
            \begin{itemize}
\pause
                \item [a] Termal Map without Trojan
\pause
                \item [b] Estimated Power Map without Trojan
\pause
                \item [c] Thermal Map with Trojan
\pause
                \item [d] Estimated Power Map with Trojan
            \end{itemize}
    \end{columns}
\end{frame}

\begin{frame}{Detection Framework}
    \begin{itemize}
        \item [*] Trojan Detection Method
            \begin{itemize}
\pause
                \item \textbf{Post-silicon Spatial Temperature}\\
                    Using infrared techniques to obtain very high resolution 
                    thermal maps by running various workloads.
\pause
                \item \textbf{Post-silicon Power Characterization}\\
                    Thermal maps are used to determine accurate and detailed
                    corresponding spatial power maps for actual trojan detection.
            \end{itemize}
    \end{itemize}
\end{frame}

\subsection{Temperature and Power Characterization}
\begin{frame}{$T\&P$ Characterization}
\pause
    \begin{itemize}
        \item \textbf{Temperature Charaterization}\\
        \begin{itemize}
            \item [*] Random Inputs
            \item [*] Estimation of power trace fo each block by Primetime-PX
            \item [*] Use HotSpot simulation tools to create steady state thermal 
                        maps of various testbench circuits.
        \end{itemize}
\pause
        \item \textbf{Power Characterization}
        \begin{equation}
            Rp + e = t
        \end{equation}
        \begin{itemize}
            \item [*] Power Characterization Formula
            \item [*] $R$ represents the thermal resistivities created by HotSpot.
            \item [*] $p$ represents the power density of the chip.
            \item [*] $e$ represents measuring noise.
            \item [*] $t$ represents the temperatures.
        \end{itemize}
    \end{itemize}
\end{frame}

\begin{frame}{Power Characterization}
    \begin{itemize}
        \item [*] Power characterization does not use the same resolution as the 
                thermal characterization. $R$ is a 10 by 10 matrix.
        \item [*] The optimization problem is to minimize the measurement error.
    \end{itemize}
\end{frame}

\section{Test Results}
\subsection{Process Variation Tests}
\begin{frame}{Process Variation Results}
        \begin{figure}[p]
            \includegraphics[width=3in]{../img/PV_thermal.PNG}
            \caption{\textbf{Detection under different porcess variation level
                    using thermal maps}}
            \label{PV_thermal}
        \end{figure}

        \begin{figure}[p]
            \includegraphics[width=3in]{../img/PV_power.PNG}
            \caption{\textbf{Detection under different porcess variation level
                    using power maps}}
            \label{PV_power}
        \end{figure}
\end{frame}

\end{document}


